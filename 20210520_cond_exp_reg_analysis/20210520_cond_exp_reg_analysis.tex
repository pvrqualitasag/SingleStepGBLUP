\documentclass[]{tufte-handout}

% ams
\usepackage{amssymb,amsmath}

\usepackage{ifxetex,ifluatex}
\usepackage{fixltx2e} % provides \textsubscript
\ifnum 0\ifxetex 1\fi\ifluatex 1\fi=0 % if pdftex
  \usepackage[T1]{fontenc}
  \usepackage[utf8]{inputenc}
\else % if luatex or xelatex
  \makeatletter
  \@ifpackageloaded{fontspec}{}{\usepackage{fontspec}}
  \makeatother
  \defaultfontfeatures{Ligatures=TeX,Scale=MatchLowercase}
  \makeatletter
  \@ifpackageloaded{soul}{
     \renewcommand\allcapsspacing[1]{{\addfontfeature{LetterSpace=15}#1}}
     \renewcommand\smallcapsspacing[1]{{\addfontfeature{LetterSpace=10}#1}}
   }{}
  \makeatother

\fi

% graphix
\usepackage{graphicx}
\setkeys{Gin}{width=\linewidth,totalheight=\textheight,keepaspectratio}

% booktabs
\usepackage{booktabs}

% url
\usepackage{url}

% hyperref
\usepackage{hyperref}

% units.
\usepackage{units}


\setcounter{secnumdepth}{2}

% citations

% pandoc syntax highlighting

% longtable
\usepackage{longtable,booktabs}

% multiplecol
\usepackage{multicol}

% strikeout
\usepackage[normalem]{ulem}

% morefloats
\usepackage{morefloats}


% tightlist macro required by pandoc >= 1.14
\providecommand{\tightlist}{%
  \setlength{\itemsep}{0pt}\setlength{\parskip}{0pt}}

% title / author / date
\title{Conditional Expectation and Regression Analysis}
\author{Peter von Rohr}
\date{2021-05-21}


\begin{document}

\maketitle



{
\setcounter{tocdepth}{2}
\tableofcontents
}

\hypertarget{disclaimer}{%
\section{Disclaimer}\label{disclaimer}}

The relationship between conditional expectation and regression analysis is summarised. The main resource was \url{https://www.le.ac.uk/users/dsgp1/COURSES/MESOMET/ECMETXT/01mesmet.pdf}.

\hypertarget{elementary-regression-analysis}{%
\section{Elementary Regression Analysis}\label{elementary-regression-analysis}}

This chapter provides a wide variety of approaches for parameter estimation ranging from least squares to maximum likelihood.

\hypertarget{general-expectation}{%
\subsection{General Expectation}\label{general-expectation}}

Let \(y\) be a continuous random variable with density \(f(y)\)\footnote{A cleaner notation would be to denote the random variable by \(\mathcal{Y}\) and the realised values by \(y\). Then the density would be \(f_{\mathcal{Y}}(y)\). But for now, we stick to the same notation as found in the above mentioned resource.}. If \(y\) is to be predicted without any further information, this can be done by its expected value \(E(y)\)\footnote{A more general definition corresponds to \(E(g(y)) = \int_y g(y) * f(y)\ dy\) for any function \(g(y)\), provided that the integral is finite.} given by

\[E(y) = \int_y y * f(y)\ dy\]

The expected value (\(E(y)\)) is the minimum-mean-square-error (mmse) predictor. If \(\pi\) is an arbitrary predictor, mean square error \(M\) is

\begin{align}
M  &=  \int_y (y - \pi)^2 f(y)\ dy \notag \\
   &=  E\{(y - \pi)^2 \} \notag \\
   &=  E(y^2) - 2\pi E(y) + \pi^2
\label{eq:mmse}   
\end{align}

The above quantity \(M\) is minimized by choosing \(\pi = E(y)\).

\hypertarget{conditional-expectation}{%
\subsection{Conditional Expectation}\label{conditional-expectation}}

Let us assume that \(y\) is statistically related to a different random variable \(x\) which was already observed. Furthermore, we assume that we know the joint density \(f(x, y)\) of \(x\) and \(y\). Then the minimum-mean-square-error predictor of \(y\) given \(x\) is given by the conditional expectation \(E(y|x)\)



\end{document}
